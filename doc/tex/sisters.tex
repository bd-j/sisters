\documentclass[manuscript, letterpaper]{aastex6}
\bibliographystyle{aasjournal}

\usepackage{graphicx}
\usepackage[suffix=]{epstopdf}
\usepackage{natbib}
\usepackage{amsmath}
\usepackage{url}
\usepackage{xspace}

\newcommand{\given}
\newcommand{\like}{\mathcal{L}}
 
\begin{document}
\author{B. Johnson}

%\begin{center}
%\today
%\end{center}

\section{Introduction}

\section{Model}
We seek to infer the parameters of a model for the age, distance, etc. distribution of stars in a cluster, using samples of the posterior distr for the individual stars.

We are going to think of the cluster as imposing a new prior on the parameters of the all the stars, 
e.g. that the ages, distances, and metallicities of all the stars be similar.
We do this via pseudo-importance sampling of heirarchical Bayesian model.
We are going to marginalize out the individual stellar parameters $\theta$.
In brief, the likelihood can be written

\begin{eqnarray}
p_i(d_i \given \phi) & = & \int d\theta \, \like(d_i | \theta) \, p(\theta \given \phi) \\
 & \approx & \sum_j p(\hat{\theta}_{i,j} \given \phi) / g(\hat{\theta_{i,j}})
\end{eqnarray}
where $d_i$ are the data for star $i$,
$\phi$ are the hyperparameters describing the cluster or ensemble of stars,
$\theta$ are the parameters for the individual star,
$\hat{theta}$ are the provided samples from the posterior PDF $p(\theta \given d_i)$,
and $g(\theta)$ is the prior used when obtaining the samples $\hat{theta}$.
Note that the second line has replaced the integral with a sum over the values of the prior at the the samples.
This is the approximate numerical integration enabled by MCMC chains.
If there is any (even an implicit) prior $g(\theta)$ that was used in the inference of the individual stellar parameters, 
that prior is inherited by the cluster model unless explicitly divided out as above.

Because division by zero is not allowed, it is impossible to remove a prior which goes to zero at any part of the parameter space.
It is also impossible to do the numeric integral over places where the chains $\theta_i$ provide no support.

To combine the constraints from all the individual stars we simply take the product of the probabilities of each star under the new prior, and muliply by the hyperprior $p(\phi)$
\begin{eqnarray}
p(\phi \given D) & \propto & p(\phi) \, \prod_i p_i(d_i \given \phi)
\end{eqnarray}
where $D \equiv {d_i}$ is the set of observations of all stars

\section{Simple Case}
In the simplest case the parameters $\phi$ might describe (narrow) gaussians in age, distance, and metallicity.

\section{Binaries}

\section{IMF}

\section{Outliers: Non-members and Wierd Stars}

\begin{eqnarray}
p_i(d_i \given \phi) & = & p_{mem} \, \sum_j p(\hat{\theta}_{i,j} \given \phi) + (1-p_{mem}) \, \sum_j p(\hat{\theta}_{i,j} \given \beta)\\
\end{eqnarray}
where $p_{mem}$ is a cluster membership probability that goes from 0 to 1,
and $\beta$ are parameters that describe the distribution of stellar properties of the non-members or background population.

The $p_{mem}$ may either be a known number for all stars, which has been inferred from other data, or an unkown number that is to be inferred from the ensmeble itself.


\section{Shrinkage}
The newly inferred priors can be used to obtain better constraints on the parameters of the individual stars.

\end{document}