\documentclass[manuscript, letterpaper]{aastex6}
\bibliographystyle{aasjournal}

\usepackage{graphicx}
\usepackage[suffix=]{epstopdf}
\usepackage{natbib}
\usepackage{amsmath}
\usepackage{url}
\usepackage{xspace}

\newcommand{\given}
\newcommand{\like}{\mathcal{L}}
\newcommand{\transpose}[1]{{#1}^{\!\mathsf T}}
\renewcommand{\det}[1]{||{#1}||}

\begin{document}
\author{B. Johnson}

%\begin{center}
%\today
%\end{center}

\begin{abstract}
Inference of cluster properties (age, metallicity, distance) is often
 accomplished by fitting isochrones to the color-magnitude diagram of member
 stars.
It can also be though of as type of heirarchical inference, where the age,
 metallicity, and age distributions of the cluster form priors for the
 individual strellar properties, and we wish to infer the parameters (e.g. mean,
 dispersion) of these priors from the properties of the stars themselves.
If the properties of the individual stars are represented as independent samples from the
 probability distributions for stellar parameters, obtained with known
 analytic priors, then this can be accomplished through pseudo importance
 weighted resampling of these chains.
We describe this technique here and present a demonstration of application to
  stars in the Hyades.
This technique has the ability to incorporate heterogenous information about
  individual stars.
It also is flexible enough to be extended to complex hyper-prior distributions
  that include multivariate dependencies.
\end{abstract}


\section{Introduction}
Discuss \match, \base9, \tausq 

\section{Model}
We seek to infer the parameters of a model for the age, distance, etc. distribution of stars in a cluster, using samples of the posterior distr for the individual stars.

We are going to think of the cluster as imposing a new prior on the parameters of the all the stars, 
e.g. that the ages, distances, and metallicities of all the stars be similar.
We do this via pseudo-importance sampling of heirarchical Bayesian model.
We are going to marginalize out the individual stellar parameters $\theta$.
In brief, the likelihood can be written

\begin{eqnarray}
p_i(d_i \given \phi) & = & \int d\theta \, \like(d_i | \theta) \, p(\theta \given \phi) \\
 & \approx & \sum_j p(\hat{\theta}_{i,j} \given \phi) / g(\hat{\theta_{i,j}})
\end{eqnarray}
where $d_i$ are the data for star $i$,
$\phi$ are the hyperparameters describing the cluster or ensemble of stars,
$\theta$ are the parameters for the individual star,
$\hat{theta}$ are the provided samples from the posterior PDF $p(\theta \given d_i)$,
and $g(\theta)$ is the prior used when obtaining the samples $\hat{theta}$.
Note that the second line has replaced the integral with a sum over the values of the prior at the the samples.
This is the approximate numerical integration enabled by MCMC chains.
If there is any (even an implicit) prior $g(\theta)$ that was used in the inference of the individual stellar parameters, 
that prior is inherited by the cluster model unless explicitly divided out as above.

Because division by zero is not allowed, it is impossible to remove a prior which goes to zero at any part of the parameter space.
It is also impossible to do the numeric integral over places where the chains $\theta_i$ provide no support.

To combine the constraints from all the individual stars we simply take the product of the probabilities of each star under the new prior, and muliply by the hyperprior $p(\phi)$
\begin{eqnarray}
p(\phi \given D) & \propto & p(\phi) \, \prod_i p_i(d_i \given \phi)
\end{eqnarray}
where $D \equiv {d_i}$ is the set of observations of all stars

\section{Simple Case}
In the simplest case the parameters $\phi$ might describe (narrow) gaussians in age, distance, and metallicity.

\section{Binaries}

\section{IMF}

\section{Outliers: Non-members and Wierd Stars}

We can use a standard mixture model. 
Here we replace the individual star probability with the following
\begin{eqnarray}
p_i(d_i \given \phi) & = & p_{mem} \, \sum_j p(\hat{\theta}_{i,j} \given \phi) + (1-p_{mem}) \, \sum_j p(\hat{\theta}_{i,j} \given \beta)\\
\end{eqnarray}
where $p_{mem}$ is a cluster membership probability that goes from 0 to 1,
and $\beta$ are parameters that describe the distribution of stellar properties of the non-members or background population.
All else stays the same

The $p_{mem}$ may either be a known number for all stars, which has been
inferred from other data, or an unkown number that is to be inferred from the ensemble itself.  
In the latter case the $p_{mem}$ of an individual star can be constructed from the posteriors.
For a posterior sample $k$ of the parameters $\phi, \beta$ the posterior odds ratio that the star $i$ is a member is
\begin{eqnarray}
p_{mem, i, k}(d_i \given \phi) & = & \frac{p_{mem, k} \, \sum_j p(\hat{\theta}_{i,j} \given \phi_k)}{(1-p_{mem, k}) \, \sum_j p(\hat{\theta}_{i,j} \given \beta_k)}\\
\end{eqnarray}



\section{Shrinkage}
The newly inferred priors can be used to obtain better constraints on the parameters of the individual stars.

\end{document}